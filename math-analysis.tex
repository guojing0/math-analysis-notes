\documentclass{article}

\usepackage{amsmath}
\usepackage{amssymb}

\title{Notes for MATH 3210: Foundation of Analysis I}
\author{Jing Guo}
\date{\today}

\begin{document}

    \maketitle
    \tableofcontents

    \section{Basic Topology}

        Metric (distance function) of $x_{0}$ and $x$: $d(x_{0}, x) = \lvert x - x_{0} \rvert$

    \subsection{Metric Space $M$}

        \begin{equation}
        d \colon M \times M \to \mathbb{R}
        \end{equation}

        \begin{enumerate}
            \item $d(x_{0}, x_{1}) \geq 0$ where $x_{0}, x_{1} \in M$
            \item $d(x_{0}, x_{1}) = 0 \iff x_{0} = x_{1}$
            \item $d(x_{0}, x_{1}) = d(x_{1}, x_{0})$ (irrespective of order)
            \item $d(x, z) \leq d(x, y) + d(y, z)$ (triangular inequity)
        \end{enumerate}

        Euclidean metric on $\mathbb{R}^{2}$:
        $d(x, y) = \sqrt{(x_{1} - y_{1})^{2} + (x_{2} - y_{2})^{2}}$

        Discrete metric: $d(x, y) = \left\{\begin{matrix}
        1 &\mbox{if}\ x \neq y \\
        0 &\mbox{if}\ x = y 
        \end{matrix}\right.$

        Open ball of radius $\epsilon$:
        $B_{\epsilon}(x_{0}) = B(x_{0}, \epsilon) = \{ y \in M \mid d(x_{0}, y) < \epsilon \}$

    \subsection{Open Sets}

        \subsubsection{Metric Spaces $X$}

        $U \subset X$ is open if for any $x \in U$ there exists $\epsilon > 0$ such that $B(x, \epsilon) \subset U$.

        Open interval is open set; closed interval is not open set.

        \subsubsection{Topological Spaces $(X, U)$}

        Let $U$ be a family of all sets, $X$ be a set. $U$ is a \textbf{topology} on $X$ if

        \begin{enumerate}
            \item $\emptyset$ (Empty set) is always open; $X$ is open. $\Leftrightarrow$ $\emptyset$ and $X$ itself belong to $U$.
            \item $F$ is a collection of open sets, then $\bigcup_{U \in F} U$ is open. $\Leftrightarrow$ Any union of members of $U$ still belongs to $U$. (Union)
            \item $F$ is a \textit{finite} collection of open sets, then $\bigcap_{U \in F} U$ is open. $\Leftrightarrow$ The intersection of any finite number of members of $U$ belongs to $U$. (Intersection)
        \end{enumerate}

        Finite case: $x \in \bigcap_{U \in F} U$, $x \in U$, hence $B(x, \epsilon_{U}) \subset U$

        \begin{gather*}
            \delta = \min \epsilon_{U} > 0 \quad where \quad U \in F \\
            B(x, \delta) \subset B(x, \epsilon_{U}) \subset U \\
            \therefore B(x, \delta) \subset \bigcap_{U \in F} U
        \end{gather*}

        Infinite case: For example, the intersection of all intervals of $( -\frac{1}{n}, \frac{1}{n})$, where $n$ is a positive number, is the set $\{ 0 \}$ which is not open in the real line.

    \subsection{Compact Sets}

        In metric space $X$, compact sets are \underline{closed}.

        Compact $\Leftrightarrow$ closed and bounded (only for Euclidean metric, $\mathbb{R}^{n}$)

\end{document}